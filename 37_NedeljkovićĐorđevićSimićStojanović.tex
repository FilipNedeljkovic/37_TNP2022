
% !TEX encoding = UTF-8 Unicode

\documentclass[a4paper]{article}

\usepackage{color}
\usepackage{url}
\usepackage[T2A]{fontenc}
\usepackage[utf8]{inputenc}
\usepackage{graphicx}

\usepackage[english,serbian]{babel}

\usepackage[unicode]{hyperref}
\hypersetup{colorlinks,citecolor=green,filecolor=green,linkcolor=blue,urlcolor=blue}

\begin{document}
	\title{Proširena stvarnost i mogućnosti njene upotrebe u obrazovanju\\ \vskip 0.1in \\ \small{Seminarski rad u oviru kursa\\ Tehničko i naučno pisanje\\ Matematički Fakultet}}
	
	\author{Autori: \\ Filip Nedeljković\\ Matija Đorđević\\ Mlađan Simić\\ Igor Stojanović}
	
	\maketitle
	
	\abstract{
		Proširena stvarnost je tehnologija koja se počela koristiti početkom 90-ih godina prošlog veka, tada u vojnoj instituciji vazdušnih snaga SAD-a. Sam termin „proširena stvarnost“ skovan je 1990. godine. Samim time, radi se o relativno novoj tehnologiji. U radu je prikazan njen razvoj, razvoj hardvera i softvera koji je ključan za rad, širu dostupnost i mogućnosti proširene stvarnosti, kako je tehnologija postala šire dostupna, način na koji se ova tehnologija može iskoristiti na šta ćemo fokus staviti na primenu tehnologije proširene stvarnosti u obrazovanju.
	}
	
	\tableofcontents
	
	\newpage
	
	\section{Uvod}
	\label{sec:Uvod}
	Proširena stvarnost (engl. Augmented reality, skraćeno AR) predstavlja takav spoj fizičkog i digitalnog sveta, u kojem digitalni elementi (slika, tekst, animacija ili zvuk) dopunjavaju fizički svet. Ljudi traže prirodniji, efikasniji i pristupačniji način za interakciju sa računarima i današnjim digitalnim svetom. Iz tog razloga okreće se tehnologijama AR i VR (engl. Virtual reality - virtuelna stvarnost). Glavna razlika između ovih tehnologija je u tome što VR zamenjuje predstavljanje stvarnog sveta, dok AR dodaje informacije stvarnom svetu, zbog čega postoji i razlika u korišćenom hardveru. Međutim, VR i AR tehnologije ne isključuju jedna drugu, naprotiv, često su komplementarne jedna drugoj, a razvoj i primena jedne je uticala na razvoj i primenu druge. AR tehnologija nam omogućava da vidimo elemente koji ne postoje u stvarnom životu putem aplikacije kroz ekran uređaja, obično mobilnog telefona.Tehnologija proširene stvarnosti (AR) se koristi u mnogim oblastima kao što su zabava, vojska, marketing, inženjering, medicina, psihologija, oglašavanje. S obzirom na naprednu tehnologiju i bogato okruženje za učenje koje nudi AR, istaknuta je i primena ove tehnologije u oblasti obrazovanja. Za razliku od AR, VR tehnologija zahteva upotrebu naočara kroz koje ne možete videti ništa oko sebe, već samo virtuelno stvoreni svet. Obe tehnologije su veoma mlade, njihov razvoj počinje u drugoj polovini 20. a šira upotreba i dostupnost tek početkom 21. veka. Ali, uprkos tome, sve brži tehnološki napredak i globalizacija doveli su do toga da se obe tehnologije veoma brzo šire, postaju dostupne sve većem broju ljudi, a svakim danom im se pronalaze nove potencijalne upotrebe.
	
\end{document}
