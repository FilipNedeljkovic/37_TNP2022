
% !TEX encoding = UTF-8 Unicode

\documentclass[a4paper]{article}

\usepackage{color}
\usepackage{url}
\usepackage[T2A]{fontenc}
\usepackage[utf8]{inputenc}
\usepackage{graphicx}

\usepackage[english,serbian]{babel}

\usepackage[unicode]{hyperref}
\hypersetup{colorlinks,citecolor=green,filecolor=green,linkcolor=blue,urlcolor=blue}

\begin{document}
	\title{Proširena stvarnost i mogućnosti njene upotrebe u obrazovanju\\ \vskip 0.2in \small{Seminarski rad u oviru kursa\\ Tehničko i naučno pisanje\\ Matematički Fakultet}}
	
	\author{Autori: \\ Filip Nedeljković\\ Matija Đorđević\\ Mlađan Simić\\ Igor Stojanović}
	\date{11.~novembar 2022.}
	\maketitle
	
	\abstract{
		Proširena stvarnost je tehnologija koja se počela koristiti početkom 90-ih godina prošlog veka, tada u vojnoj instituciji vazdušnih snaga SAD-a. 
		Sam termin „proširena stvarnost“ skovan je 1990. godine. Samim time, radi se o relativno novoj tehnologiji. U radu je prikazan njen razvoj, 
		razvoj hardvera i softvera koji je ključan za rad, širu dostupnost i mogućnosti proširene stvarnosti, kako je tehnologija postala šire dostupna, 
		način na koji se ova tehnologija može iskoristiti na šta ćemo fokus staviti na primenu tehnologije proširene stvarnosti u obrazovanju.
	}
	
	\tableofcontents
	
	\newpage
	
	\section{Uvod}
	\label{sec:Uvod}
	Proširena stvarnost (engl. Augmented reality, skraćeno AR) predstavlja takav spoj fizičkog i digitalnog sveta, u kojem digitalni elementi (slika, tekst, 
	animacija ili zvuk) dopunjavaju fizički svet. Ljudi traže prirodniji, efikasniji i pristupačniji način za interakciju sa računarima i današnjim digitalnim 
	svetom. Iz tog razloga okreće se tehnologijama AR i VR (engl. Virtual reality - virtuelna stvarnost). Glavna razlika između ovih tehnologija je u tome što VR 
	zamenjuje predstavljanje stvarnog sveta, dok AR dodaje informacije stvarnom svetu, zbog čega postoji i razlika u korišćenom hardveru. Međutim, VR i AR 
	tehnologije ne isključuju jedna drugu, naprotiv, često su komplementarne jedna drugoj, a razvoj i primena jedne je uticala na razvoj i primenu druge. AR 
	tehnologija nam omogućava da vidimo elemente koji ne postoje u stvarnom životu putem aplikacije kroz ekran uređaja, obično mobilnog telefona.Tehnologija 
	proširene stvarnosti (AR) se koristi u mnogim oblastima kao što su zabava, vojska, marketing, inženjering, medicina, psihologija, oglašavanje. S obzirom na 
	naprednu tehnologiju i bogato okruženje za učenje koje nudi AR, istaknuta je i primena ove tehnologije u oblasti obrazovanja. Za razliku od AR, VR tehnologija 
	zahteva upotrebu naočara kroz koje ne možete videti ništa oko sebe, već samo virtuelno stvoreni svet. Obe tehnologije su veoma mlade, njihov razvoj počinje u 
	drugoj polovini 20. a šira upotreba i dostupnost tek početkom 21. veka. Ali, uprkos tome, sve brži tehnološki napredak i globalizacija doveli su do toga da se 
	obe tehnologije veoma brzo šire, postaju dostupne sve većem broju ljudi, a svakim danom im se pronalaze nove potencijalne upotrebe.
	
	\section{Primena}
	\label{sec:Primena}
	Obrazovni sistem evoluira i prilagođava se dostupnoj tehnologiji i potrebama studenata. Tradicionalne knjige zamenjuju se digitalnim nastavnim sadržajima. 
	U većini akademskih oblasti, kao što su matematika, nauka, inženjerstvo i statistika, uspeh u ime studenta u velikoj meri zavisi od njegove sposobnosti da 
	predvidi i manipuliše apstraktnim informacijama. Jedan od glavnih razloga zašto se VR i AR koriste u svrhe obrazovanja i obuke je podrška visoke interaktivnosti 
	i sposobnosti predstavljanja virtuelnog okruženja koje podseća na stvarni svet. Pomoću ove tehnologije moguće je istraživanje i manipulisanje trodimenzionalnim 
	interaktivnim okruženjem. Tradicionalne metode obrazovanja zasnovane na predavanjima suočavaju se sa problemom nedovoljne  angažovanosti studenata, što za posledicu 
	može da ima negativan uticaj na njihov uspeh.


	Istraživači potvrđuju da tehnološki alati koji se koriste u obrazovanju doprinose aktiviranju motivacije, nude nove mogućnosti, pružaju prijatnu atmosferu za učenje, 
	povećavaju interakciju među učenicima, čine proces učenja aktivnijim, efikasnijim i sadržajnijim. Na sličan način AR tehnologija privlači pažnju naučnika u obrazovanju 
	jer favorizuje interakciju stvarnih i virtuelnih objekata i stvara okruženje koje podstiče iskustveno učenje. Naučnici smatraju da AR tehnologija pomaže u predstavljanju 
	apstraktnih stvari, u prikazivanju opasnih slučajeva, prikazivanju složenih tema, zatim u podučavanju nevidljivih stvari i događaja. Štaviše, AR tehnologija pruža studentima 
	fleksibilnost, podstiče veštine kreativnog razmišljanja, veštine tumačenja i rešavanja problema. Mnogi naučnici koji istražuju upotrebu tehnologije i interneta u obrazovanju 
	ukazuju na značaj karakteristika učenika i upotrebe tehnologije za poboljšanje ishoda učenja u oblasti visokog obrazovanja.

	Construct3D omogućava studentima da uče koncepte mašinstva, matematike ili geometrije. Nastavni sadržaji iz matematike koji se odnose na tri dimenzije mogu biti teški za učenike 
	jer moraju stalno u glavi da pretvaraju dvodimenzionalni crtež iz udžbenika u trodimenzionalni model i obrnuto. Zbog toga učenici imaju dodatni izazov u kome treba da steknu 
	trodimenzionalna znanja iz dvodimenzionalnih materijala. Trenutno se ovaj problem donekle rešava uz pomoć fizičkih modela geometrijskih tela i 3D grafičkih kalkulatora (npr. GeoGebra 3D kalkulator).
	
	AR aplikacije za hemiju omogućavaju studentima da vizuelizuju i komuniciraju sa prostornom strukturom molekula koristeći ručni marker. HP Reveal, besplatna aplikacija, koristi 
	se za kreiranje AR kartica za proučavanje mehanizama organske hemije i za kreiranje virtuelnih demonstracija kako se koriste laboratorijski instrumenti.

	Deca u osnovnoj školi lako uče iz interaktivnih iskustava. Astronomska sazvežđa i kretanje objekata u Sunčevom sistemu mogu se prikazati u pravcu u kome se uređaj drži
	i proširiti dodatnim video informacijama. Ilustracije naučnih knjiga na papiru mogu da zažive u vidu video zapisa bez potrebe da učenik traži video materijale na internetu.

	Proširena stvarnost se može koristiti i kao pomoćno sredstvo u radu sa učenicima koji imaju smetnje u razvoju. Jedan mogući okvir učenja koji koristi proširenu stvarnost 
	i gamifikaciju koja nudi dodatnu podršku učenicima sa intelektualnim teškoćama u procesu učenja je da učenici grupišu predmete koji pripadaju životinjama ili voću. 
	Učenici biraju oznake sa nazivima i postavljaju ih ispred kamera, nakon čega se na računaru kreira slika izabrane životinje/voća i računar izgovara naziv životinje/voća.


	\section{Zaključak}
	\label{sec:Zaključak}
	U okruženju koje ima karakteristike globalizovanog, dinamičnog i visoko \hyphenation{konku-re-tnog} konkuretnog, prihvatanje i implementacija digitalnih 
	tehnologija postaje neminovna. U takvim uslovima obrazovne institucije suočavaju se sa potrebom da uvrste iste u svoj rad. VR i AR se sve više primenjuje
	u okviru obrazovnih procesa i njihov potencijal umnogome doprinosi studentima. U radu je istaknuto nekoliko od brojnih prednosti koje tehnologija koja prevodi
	čoveka u virtuelni svet nosi sa sobom. Po ugledu na primer poznatih svetskih univerziteta i njihovo široko usvajanje virtuelne i proširene stvarnosti, prepoznat 
	je značaj praćenja trendova u obrazovnoj tehnologiji. Upotreba VR i AR tehnologije i efekti te upotrebe veoma su raznoliki, ali ono što se izdvaja kao zajedničko 
	jeste mogućnost neposredne interakcije sa fizički nedostupnim objektima i upoznavanje sa predmetima i situacijama na interesantan i razumljiv način.

\end{document}
